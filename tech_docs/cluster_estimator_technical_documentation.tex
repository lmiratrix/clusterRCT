\documentclass[]{article}
\usepackage{amsmath,amssymb}
\usepackage{lmodern}
\usepackage{iftex}
% Options for packages loaded elsewhere
\PassOptionsToPackage{unicode}{hyperref}
\PassOptionsToPackage{hyphens}{url}
%
\ifPDFTeX
  \usepackage[T1]{fontenc}
  \usepackage[utf8]{inputenc}
  \usepackage{textcomp} % provide euro and other symbols
\else % if luatex or xetex
  \usepackage{unicode-math}
  \defaultfontfeatures{Scale=MatchLowercase}
  \defaultfontfeatures[\rmfamily]{Ligatures=TeX,Scale=1}
\fi
% Use upquote if available, for straight quotes in verbatim environments
\IfFileExists{upquote.sty}{\usepackage{upquote}}{}
\IfFileExists{microtype.sty}{% use microtype if available
  \usepackage[]{microtype}
  \UseMicrotypeSet[protrusion]{basicmath} % disable protrusion for tt fonts
}{}
\makeatletter
\@ifundefined{KOMAClassName}{% if non-KOMA class
  \IfFileExists{parskip.sty}{%
    \usepackage{parskip}
  }{% else
    \setlength{\parindent}{0pt}
    \setlength{\parskip}{6pt plus 2pt minus 1pt}}
}{% if KOMA class
  \KOMAoptions{parskip=half}}
\makeatother
\usepackage{xcolor}
\usepackage[margin=1in]{geometry}
\usepackage{graphicx}
\makeatletter
\def\maxwidth{\ifdim\Gin@nat@width>\linewidth\linewidth\else\Gin@nat@width\fi}
\def\maxheight{\ifdim\Gin@nat@height>\textheight\textheight\else\Gin@nat@height\fi}
\makeatother
% Scale images if necessary, so that they will not overflow the page
% margins by default, and it is still possible to overwrite the defaults
% using explicit options in \includegraphics[width, height, ...]{}
\setkeys{Gin}{width=\maxwidth,height=\maxheight,keepaspectratio}
% Set default figure placement to htbp
\makeatletter
\def\fps@figure{htbp}
\makeatother
\setlength{\emergencystretch}{3em} % prevent overfull lines
\providecommand{\tightlist}{%
  \setlength{\itemsep}{0pt}\setlength{\parskip}{0pt}}
\setcounter{secnumdepth}{-\maxdimen} % remove section numbering
\ifLuaTeX
  \usepackage{selnolig}  % disable illegal ligatures
\fi
\IfFileExists{bookmark.sty}{\usepackage{bookmark}}{\usepackage{hyperref}}
\IfFileExists{xurl.sty}{\usepackage{xurl}}{} % add URL line breaks if available
\urlstyle{same} % disable monospaced font for URLs
\hypersetup{
  pdftitle={Cluster RCT Estimators Technical Document},
  pdfauthor={Miratrix},
  hidelinks,
  pdfcreator={LaTeX via pandoc}}






\title{Cluster RCT Estimators Technical Document}
\author{Miratrix}
\date{2022-11-16}

\begin{document}
\maketitle

\section{Introduction}\label{introduction}

This gives technical details for the identified estimators for
estimating average effects for Blocked, Cluster-Randomized RCTs.

Once an estimand is identified, a researcher must decide how to estimate
it, given the observed data, and obtain uncertainty estimates (standard
errors) for those estimates. We call the ATE estimators ``effect
estimators'' and the uncertainty estimators ``standard error
estimators.'' In this section, we systematically go through three
general classes of effect estimators, discussing variants of
implementing each. As in our work focused on multisite randomized
trials, we divide our estimators into three general classes of linear
modeling estimators, multilevel (random effects) estimators, and
design-based estimators.

Because blocking is so prevalent for cluster randomized trials, we
proceed with this full blocking incorporated in our discussion, even
though it increases some of the technical detail and complexity. When
there is no blocking, one can simply assume a single block, holding all
the clusters, in most cases. We will note departures from this as we go
along.

Our notation is to discuss individual students ($i$) in clusters
(e.g., schools) $j$, nested in blocks (e.g., districts or sites) $k$.


Before we look at estimators, let's spend a moment thinking about estimands.
In particular, for each individual we have $Y_{ijk}(0)$ and $Y_{ijk}(1)$, with an individual treatment effect of $\tau_{ijk} = Y_{ijk}(1) - Y_{ijk}(0)$.
We can never see both potential outcomes, and thus will never know $\tau_{ijk}$ for any individual, regardless of how we assign treatment.

Similarly, for each cluster we have the average treatment effect in that cluster as
$$
\beta_{jk} = \frac{1}{N_{jk}} \sum_{i=1}^{N_{jk}} Y_{ijk}(1) - Y_{ijk}(0) .
$$
The $\beta_{jk}$ are the average treatment effect of the individuals in cluster $jk$.
Similar to the $\tau_{ijk}$, we have no ability to estimate a cluster's $\beta_{jk}$ because we treat the entire cluster, or not.
This is very different from multisite randomized trials, where we get to see some of the individuals in a site treated and some not treated, allowing for (a perhaps noisy) estimate of the site's average treatment impact.

Because of this, a cluster-randomized experiment can be thought of, in a sense, as an individually randomized experiment where the individuals are the clusters themselves.
In fact, this view is prevalent in much of the design based causal inference literature.
It is also the intuition why we might think of our very large (in terms of $N$) experiment as quite small (in terms of $J$, the total number of clusters).

Within a district $k$ we have different possible estimands of interest.
Similar to Miratrix, Weiss and Henderson (CITE), we can have the simple average of the $\beta_{jk}$ (the cluster-average effect) of
$$ \beta_k = \sum_j \beta_{jk} / J_k , $$
or the person-weighted average of the $\beta_{jk}$ (the person-average effect) of
$$ \beta_k = \frac{1}{N_k} \sum_j N_{jk} \beta_{jk} .$$
The latter is the average effect across all individuals in site $k$.
The former is the average effect of the clusters, regardless of their size, in site $k$.

We can then weight these $\beta_k$ across the districts in different ways: a simple average, giving the district-average of the district-level estimands, regardless of district size, a cluster-weighted average, weighing each district by the number of clusters in it, or a person-weighted average, weighing each district by the number of people in it.

We argue it is not typically sensible to, for example, calculate a person-weighted average of cluster-average impacts.
The weighting at level 3 should be aligned with the weighting at level 2.
Furthermore, we generally think it is a rare instance where a simple average across the districts is a desired quantity.
That all being said, there are many weighting options one might select.



\subsection{Acknowledgements}\label{acknowledgements}

Parts of this document, in particular some notation and mathematical
formulation, are taken from the technical supplement to the Hunter et
al.~paper on PUMP.



\section{Linear Regression Estimators}\label{linear-regression-estimators}

One of the most common estimation strategies for cluster RCTs in the
social sciences is linear regression. For a two-level context (no
districts), a natural choice is to fit a linear regression of the
outcome onto the treatment indicator (and some individual or
cluster-level control variables) and then use cluster robust standard
errors (CRVE).

Here the individual nature of the data will naturally target a
person-average estimand, as larger clusters have larger weight. We
cannot have fixed effects at the cluster level, as these would be
completely confounded with treatment assignment, but we can have them at
the district level.

We give some notation to connect to other regression estimators
discussed below. Define $K$ site dummies $S_{q,ijk}$,
$q = 1, \ldots, K$ with $S_{q,ijk} = 1$ if individual $ijk$ is in
site $k$; in other words, $S_{q,ijk} = 1\{ q = k \}$.

The simplest OLS working model for estimating an impact would then
arguably be

\[
\begin{aligned}
Y_{ijk} &=  \sum_{q=1}^K \alpha_{q} S_{q,ijk} + \beta T_{ij} + \epsilon_{ij} ,
\end{aligned}
\] where the $\alpha_q$ are the $K$ fixed effects for the districts.
This model has $K+1$ coefficients in total. The covariance of the
errors would then be a block-diagonal matrix, one block per cluster (not
district). We exclude an overall intercept to have individual fixed
effects for each district and no reference district.

One advantage of the fixed effects model compared to design-based is the
single impact parameter reduces the degrees of freedom used by the
model, which could improve asymptotic inference in smaller studies.
Given how few clusters exist in many education experiments, this may
indeed be worthwhile.

For simplicity, we can sometimes leave the dummy variables implicit,
simply writing \[
 Y_{ijk} =  \alpha_k + \beta T_{ijk} + \epsilon_{ijk}.
\]

\subsection{OLS interacted by district}\label{ols-interacted-by-district}

We can extend the above model to estimate impacts within each district
as so:

\[
\begin{aligned}
Y_{ijk} &= \sum_{q=1}^K \alpha_{q} S_{q,ijk} + \sum_{q=1}^K \beta_k  S_{q,ijk} T_{ij} + \epsilon_{ij} \\
 &= \alpha_k + \beta_k T_{ijk} + \epsilon_{ijk} .
\end{aligned}
\]

Now we have treatment-by-district interaction terms. This model will
estimate some sort of average within each district, giving $K$
estimates, $\hat{\beta}_k$, $k=1,\ldots,K$, in total.

To get an overall estimate we would then average these. This opens the
door to needing to make explicit choices about weighting. For example if
we are wanting to weight each cluster equally, we would weight each
district by the number of clusters it has, as follows:

\begin{align}
\hat{\beta} = \sum_{k=1}^K \frac{J_k}{J} \hat{\beta}_{k}, \label{eq:cluster_weight}
\end{align}
where $J_k$ are the number of clusters in district $k$.

If we wanted to tend towards the individual-average effect we would weight differently. E.g., we might weight as
\[
\hat{\beta} = \sum_{k=1}^K \frac{N_k}{N} \hat{\beta}_{k} ,
\] where $N_k$ are the number of individuals in district $k$.

Both of these weighting schemes are somewhat suspect: we are weighting the $\hat{\beta}_k$, but what are the $\hat{\beta}_k$ themselves estimating?
If they are estimating the average treatment impact for the individuals in site $k$, then the cluster-weighted expression (Equation \ref{eq:cluster_weight}) is going to weight each district by the number of clusters, but each district estimate is the person-average treatment effect.
The final implied estimand is not easily considered.
If we truly want the average cluster impact, we would need to ensure the $\hat{\beta}_k$ are aligned with our final district weighting step.


For our linear regression model, we argue that the $\hat{\beta}_k$ are estimating 




\section{Aggregation Estimators}\label{aggregation-estimators}

Another regression approach is to first aggregate the data within each
cluster to obtain the average cluster outcome ($\bar{Y}_{jk}$),
defined in terms of the individual potential outcomes as

\[ \bar{Y}_{jk} = \frac{1}{N_{jk}}  \sum_{i=1}^{N_{jk}} Y_{ij} = Z_{jk} \bar{Y}_{jk}(1) + (1-Z_{jk})\bar{Y}_{jk}(0) \]

Here $Z_{jk}$ is an 0/1 treatment indicator for cluster j, and we
observe either the average of the cluster's treatment or control
potential outcomes. We then run a subsequent regression of the
aggregated outcomes onto their treatment indicators. This approach
collapses the individual-level data to get measures at the level of
randomization. We can then weight these averages differently for
different estimands. If cluster sizes differ, we will also potentially
have to account for heteroskedasticity, where some estimated cluster
averages (the average of $N_{jk}$ individuals) will be more uncertain
(due to smaller samples of individuals) than others. In this case, we
can use heteroskedastic robust standard errors to account for this. See
Schochet (2020) for more.

When we have a blocking variable, there are a few options. First, one
can simply include district-level fixed effects as above. Alternatively,
one can include district by treatment interactions, allowing different
districts to have different average ATEs, and then these district ATEs
can be aggregated to estimate the overall ATE. These can give different
weighted averages of impacts across units. As part of this work, we will
identify which specific estimands are being targeted by all these
different choices.

\section{Hierarchical linear models (aka random effects models)}\label{hierarchical-linear-models-aka-random-effects-models}

Multilevel modeling explicitly captures the hierarchical nature of the
data and allows for estimation of the degree of variation in outcomes at
each level (and variation in impacts in the three-level case) as well as
the overall ATE. Multilevel modeling is also probably the most common
tool used to analyze data from CRT experiments in education contexts. As
a hint as to typical practice, we discuss the models identified in the
popular PowerUp! power calculator (Dong \& Maynard, 2013). This tool
explicitly notes different couplings of design and modeling choices, and
provides specific uncertainty formulas for several contexts.

For the two-level case, PowerUp! suggests a random intercept model, with
a single coefficient for average treatment effect. This assumes a
constant treatment effect for all schools, and we can also include
school covariates for the intercept.

The model for estimating impacts on outcome $m$ is given by:

\[
\begin{aligned}
Y_{ijk} &=  \theta_{0,jk} + \gamma_{mp}^{\top} C_{ijkp} + r_{ijk}\\
\theta_{0,jk} &= \mu_{0,k} + \psi_{1} T_{jk} + \sum_{r=1}^{g_{2}} \delta_{mr} X_{jkr} + u_{0,jk}
\end{aligned}
\]

with reduced form:

\[
Y_{ijk} = \psi_{1} T_{jk} + \mu_{0,k} + \sum_{r=1}^{g_{2}} \delta_{mr} X_{jkr} + \sum_{p=1}^{g_{1}} \gamma_{mp} C_{ijkp} + u_{0,jk} + r_{ijk}
\]

and distributions: \[
\begin{aligned}
u_{0,jk} &\sim N\left(0, \tau^2_{0}\right)\\
r_{ijk} &\sim N\left(0, \sigma^2_m\right).
\end{aligned}
\]

The standard error of the treatment effect estimate is given by: \[
Q_m = \sqrt{\frac{\text{ICC}_{2}(1 - R^2_{2})}{\bar{T}(1 - \bar{T}) J} + \frac{(1-\text{ICC}_{2})(1-R^2_{1})}{\bar{T}(1 - \bar{T}) J \bar{n}}}.
\] The degrees of freedom are \[
\text{df}_m = J - g_{1} - 2.
\]

In R, fitting this multilevel model would be have a \texttt{lmer()}
formula along the lines of:

\begin{verbatim}
Yobs ~ 1 + T.x + X.jk + C.ijk + (1 | S.id)
\end{verbatim}

The constant effects model means that we assume no treatment variation
across our sites, i.e., that $\omega_{2} = 0$. The resulting impact
estimate when there is impact heterogeneity across clusters is likely to
be a form of precision weighting.

The modeling options multiply once we allow for three levels, as we then
can potentially model cross-site means and impact variation using either
fixed or random effects at level 3. PowerUp! identifies two estimators
for this context, which they call fixed and random effects. The random
effect model allows the treatment coefficient to vary across sites
(implying a superpopulation of sites). We discuss these two in the
following

\hypertarget{fixed-blocks-random-effects-for-clusters}{%
\subsection{Fixed blocks, random effects for
clusters}\label{fixed-blocks-random-effects-for-clusters}}

In this model we have fixed intercepts for districts, fixed treatment
effects for districts, random intercepts for schools, constant effects
for schools within a district, school covariates for intercept.

The model for estimating impacts on outcome $m$ is given by: \[
\begin{aligned}
Y_{ijk} &=  \theta_{0,jk} + \sum_{p=1}^{g_{1}} \gamma_{mp} C_{ijkmp} + r_{ijk}\\
 \theta_{0,jk} &= \mu_{0,k} + \psi_{1,jk} T_{jk} + \sum_{r=1}^{g_{2}} \delta_{mr} X_{jkmr} + u_{0,jk}\\
 \mu_{0,k}  &= \Xi_{0}  + w_{0,k}\\
 \psi_{1,k} &= \Xi_{1} + w_{1,k}
\end{aligned}
\] with reduced form: \[
Y_{ijk} = \left(\Xi_{1} + w_{1,k} \right) T_{jk} + \Xi_{0} + \sum_{r=1}^{g_{2}} \delta_{mr} X_{jkmr} + \sum_{p=1}^{g_{1}} \gamma_{mp} C_{ijkmp}+ w_{0,k} + u_{0,jk} + r_{ijk}
\]

and distributions: \[
\begin{aligned}
u_{0,jk} &\sim N\left(0, \tau^2_{0}\right)\\
r_{ijk} &\sim N\left(0, \sigma^2_m\right).
\end{aligned}
\]

The standard error of the treatment effect estimate is given by: \[
Q_m = \sqrt{
\frac{\text{ICC}_{2}(1 - R^2_{2})}{\bar{T}(1 - \bar{T}) JK} +
\frac{(1-\text{ICC}_{2} - \text{ICC}_{3})(1-R^2_{1})}{\bar{T}(1 - \bar{T}) J K\bar{n}} }.
\]

The degrees of freedom are \[
\text{df}_m = K( J - 2) - g_{2}.
\]

This model assumes no variation of impacts within schools, and no
variation at the district level, i.e., that $\omega_2 = 0$.

The R model is

\begin{verbatim}
Yobs ~ 0 + T.x * D.id - T.x + X.jk + C.ijk + (1 | S.id)
\end{verbatim}

The overall treatment effect is then the average of the T.x interaction
terms.

\subsection{Full random effects model}\label{full-random-effects-model}

In this model we have random intercepts and impact effects for
districts, random intercepts for schools, constant effects for schools
within a district, school and district covariates for intercept. Powerup
also allows for district covariates for treatment effects.

The model for estimating impacts on outcome $m$ is given by: \[
\begin{aligned}
Y_{ijk} &=  \theta_{0,jk} + \sum_{p=1}^{g_{1}} \gamma_{mp} C_{ijkmp} + r_{ijk}\\
 \theta_{0,jk} &= \mu_{0,k} + \sum_{r=1}^{g_{2}} \delta_{mr} X_{jkmr} + u_{0,jk}\\
 \mu_{0,k}  &= \Xi_{0} + \psi_{1,k} T_{k} + \sum_{s=1}^{g_{3}} \xi_{ms} V_{kms} + w_{0,k} \\
 \psi_{1,jk} &= \Xi_{1} + w_{1,k}
\end{aligned}
\] with reduced form: \[
\begin{aligned}
Y_{ijk} &= \left(\Xi_{1} + w_{1,k}\right) T_{jk} + \Xi_{0}\\
 & + \sum_{s=1}^{g_{3}} \xi_{ms} V_{kms} + \sum_{r=1}^{g_{2}} \delta_{mr} X_{jkmr} + \sum_{p=1}^{g_{1}} \gamma_{mp} C_{ijkmp}\\
 &+ w_{0,k} + u_{0,jk} + r_{ijk}
\end{aligned}
\] and distributions: \[
\begin{aligned}
u_{0,jk} &\sim N\left(0, \tau^2_{0}\right)\\
\nonumber \begin{pmatrix} w_{0, k} \\ w_{1,k}\\ \end{pmatrix} &\sim
N\left(\begin{pmatrix} 0 \\ 0\\ \end{pmatrix}, \begin{pmatrix} \eta^2_{0} & \kappa^w_{m} \eta_{0} \eta_{1} \\ \kappa^w_{m} \eta_{1} \eta_{0} & \eta^2_{1} \\ \end{pmatrix}\right) \\
\nonumber r_{ijk} &\sim N\left(0, \sigma^2_m\right).
\end{aligned}
\]

Similar to other blocked models model, in PowerUp! they further assume
that district covariates also influence the treatment impact \[
\mu_{1,jk} = \xi_{1} + \sum_{s=1}^{g_{3}} \zeta_{mr} V_{kms} w_{1,k}
\] but we do not make this assumption.

The standard error of the treatment effect estimate is given by: \[
\begin{aligned}
Q_m = \sqrt{
\frac{\text{ICC}_{3} \omega_{3}}{K} +
\frac{\text{ICC}_{2}(1 - R^2_{2})}{\bar{T}(1 - \bar{T}) J K } +
\frac{(1-\text{ICC}_{2} - \text{ICC}_{3})(1-R^2_{1})}{\bar{T}(1 - \bar{T}) J K\bar{n}} }.\end{aligned}
\]

The degrees of freedom are often taken as $\text{df}_m = K - 1$. We
usually further assume $\omega_{2} = 0$.

The R model formula for the above is

\begin{verbatim}
Yobs ~ 1 + T.x + V.k + X.jk + C.ijk + (1 | S.id) + (1 + T.x | D.id)
\end{verbatim}

\subsection{Discussion}\label{discussion}

The fixed effect model puts no distribution on the u, but still has the
random effect at level 2 for the intercept of each cluster within the
sites (implying a finite sample of sites). There are other modeling
options within this framing, such as an extension of the FIRC model
(Bloom et al., 2017) with only a distribution on $u_1k$.






\section{Design-based estimators}\label{design-based-estimators}

Design-based estimators, although not as common as linear regression or
multi-level modeling, are usually directly and explicitly tied to an
estimand of interest (Schochet, 2015). Design-based estimators focus on
the random assignment mechanisms and the sampling mechanism as the
sources of uncertainty, rather than specifying linear models with random
residual components. They are generally considered to rely on weaker
assumptions than other approaches; e.g., homoskedasticity assumptions,
or assumptions of a random normal residual, are not needed.

Design-based estimators work with the cluster averages (sometimes
adjusted with covariates). We first focus on a single site $k$.
The simplest cluster-average estimator for site $k$ would be

\[
\hat{\beta}_{k-cluster} = \frac{1}{J_{k1}}  \sum_{j=1}^{J_k} Z_{jk} \hat{Y}_{jk} - \frac{1}{J_{k0}} \sum_{j=1}^{J_k} (1-Z_{jk}) \hat{Y}_{jk} ,
\]
where \$J\_\{k1\} and $J_{k0}$ are the number of clusters assigned to treatment or control status, respectively.
The $\hat{Y}_{jk}$ are the observed mean outcomes of the clusters.
This is a version of the
aggregation approach, above.
This estimator targets the average of the cluster average effects, in
site $k$.


For district $k$, the person-average estimator would be \[
\hat{\beta}_{k-person} = \frac{1}{N_{k1}} \sum_{j=1}^{J_k} Z_{jk} N_{jk} \widehat{Y}_{jk} - \frac{1}{N_{k0}} \sum_{j=1}^{J_k} (1-Z_{jk}) N_{jk} \hat{Y}_{jk} 
\]

where $N_{k1} = \sum Z_{jk} N_{jk}$ are the number of individuals
assigned to treatment, and $N_{k0}$ is defined similarly. A
site-size-weighted average of these gives the effect for the average
person: For the site-average of cluster averages, we weight differently
again: \[
\beta = \sum_{k=1}^K \frac{1}{K} \beta_{k-person} 
\] There are even more weightings options possible, such as the
district-average of the cluster-average impact.



Uncertainty estimation depends on the population model used. For
example, we could focus entirely on the randomization that was
conducted, holding individuals, clusters, and sites as fixed (i.e.,
finite). This fully finite-sample view is the most well-known for
design-based inference. Here we use design-based uncertainty estimators
within each site (with weights to align with the ATE estimator), and
then aggregate based on site size, again using weights depending on our
target estimand. For this general context, see Middleton and Aronow
(2015) or Schochet (2015).

Alternatively, we could model the clusters in each site as sampled from
infinite populations at each site. Students could then be finite, or
not, within this framing. If we view the clusters as sampled from a
fixed set of superpopulations, one for each site, with individuals
fixed, we can use Schochet et al.~(2021) for uncertainty estimation.
Other models are possible; see Pashley and Miratrix (2021a, 2021b) for
further discussion in the context of individual randomization with
blocking. We will identify the different estimators for uncertainty
estimation in the literature and tie them to their specific estimands
and sampling frameworks.




\section{Covariate adjustment}\label{covariate-adjustment}

Cluster randomized experiments are notoriously difficult to power. To
improve precision researchers frequently turn to level 1 and level 2
covariates to explain some of the outcome variation and thus reduce
standard errors of an average effect estimator. While we focused on the
no-covariate case above for clarity, the covariate-adjusted models
parallel that discussion.

Most of the various estimation strategies, in particular the linear
regression models, explicitly allow for covariate adjustment. The
design-based estimators have covariate-based sister estimators that can
be shown to generalize the special no-covariate case. E.g., Shochet et
al (2020) and (2021) show how design-based approaches, rooted in
randomization, can be expressed as a weighted linear regression, which
naturally then allows for covariate adjustment. We use this work that
shows equivalence to use weighted regression for our design based
estimators.

\end{document}
